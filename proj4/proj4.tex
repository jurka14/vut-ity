\documentclass[a4paper, 11pt]{article}

\usepackage[czech]{babel}
\usepackage[utf8]{inputenc}
\usepackage[left=2cm, top=3cm, text={17cm, 24cm}]{geometry}
\usepackage{times}
\usepackage[unicode]{hyperref}
\hypersetup{colorlinks = true, hypertexnames = false}

\begin{document}

\begin{titlepage}
		\begin{center}
			\Huge
			\textsc{Vysoké učení technické v Brně} \\
			\huge
			\textsc{Fakulta informačních technologií} \\
			\vspace{\stretch{0.382}}
			\LARGE
			Typografie a publikování\,--\,4. projekt \\
			\Huge
			Bibliograf{}ické citace
			\vspace{\stretch{0.618}}
		\end{center}

		{\Large
			\today
			\hfill
			Vojtěch Jurka
		}
	\end{titlepage}
	
	

\section{\LaTeX}
\LaTeX\ je balík maker programu \TeX, který umožňuje autorům sázet díla ve vysoké typografické kvalitě. Jeho aktuální verze je označena \LaTeX\ 2$\epsilon$. Byl původně napsán Lesliem Lamportem, který o něm napsal i knihu \cite{lamport}. \LaTeX\ užívá programu \TeX\ jako sázecího stroje. Pro více základních a stručných informací o tomto programu, viz \cite{wiki} a \cite{martinek}.

\subsection{Používání \LaTeX u}
Práce s \LaTeX em může laikovi připomínat programování, ačkoli k němu má celkem daleko. Je to tak kvůli stylu práce s tímto programem, jež sestává ze tří kroků, kterými jsou psaní a úprava zdrojového textu, jeho následný překlad, při němž probíhá sázení a konečným prohlížením vysázeného dokumentu. O bližších informacích o~překladu a sázení pojednává \cite{rybicka}.

\subsection{Dělení dokumentu}
Dokument se v \LaTeX u dělí na 2 části. Tou první je hlavička, neboli preambule. Druhou částí je už samotný text. Více informací - \cite{root}.

\subsection{Matematická sazba}
Sazba jakýchkoli vzorců a matematiky celkově je hlavní doménou \LaTeX u. Pro sázení matematických textů a vzorců jsou k dispozici různá prostředí, jak se můžeme dočíst v \cite{math}. Zde máme několik příkladů vysázených matematických rovnic:

$$
l = \frac{2b^2}{a} = 2a(e^2-1)
$$

$$
\frac{(\hat{l} + \hat{i} + \hat{k})}{\sqrt{3}}
$$

$$
\frac{x^2}{b^2} + \frac{y^2}{a^2} = 1
$$


Tyto a podobné rovnice lze nalézt v publikacích jako je \cite{series}, \cite{frac} a \cite{matikacz}.

\subsection{Výhody a nevýhody \LaTeX u}
\LaTeX má mnoho výhod, kvůli kterým je dobré ho používat. Ty hlavní jsou typografická preciznost, kterou disponuje a také již zmiňovaná velice vyvinutá sazba matematiky. Jeho hlavní nevýhodou je  počáteční potřeba se ho naučit používat, hlavně jeho syntax. Není tak intuitivní jako běžné textové editory. Přináší to s sebou střet dvou potřeb. \LaTeX\ je vhodnější pro tvorbu delších, strukturově a sazbově složitějších dokumentů, zatímco běžné grafické textové editory jsou vhodnější pro krátké a jednoduché dokumenty díky své jednoduchosti. Jeho další nevýhodou je fakt, že není tak rozšířený a na většině zařízení nainstalovaný není. Naproti tomu je ale dostupný pro většinu operačních systémů, a například narozdíl od Wordu, je zcela zdarma, jak uvádí \cite{vyhody}.

\newpage
	\bibliographystyle{czechiso}
	\renewcommand{\refname}{Literatura}
	\bibliography{proj4}

\end{document}
