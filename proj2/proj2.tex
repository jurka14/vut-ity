\documentclass[11pt, a4paper]{article}

\usepackage[utf8]{inputenc}
\usepackage[left=1.5cm,text={18cm, 25cm},top=2.5cm]{geometry}
\usepackage[IL2]{fontenc}
\usepackage[czech]{babel}
\usepackage{verbatim}
\usepackage[unicode]{hyperref}
\usepackage{times}
\usepackage{amsthm}
\usepackage{amsmath}
\usepackage{amsfonts}

\newtheorem{theorem}{Věta}

\theoremstyle{definition}
\newtheorem{definition}{Definice}


\begin{document}
\thispagestyle{empty}
\begin{center}
{\Huge\textsc{Fakulta informačních technologíí \\ Vysoké učení techniké v Brně}\\}

\vspace{\stretch{0.382}}

{\LARGE Typografie a publikování – 2. projekt \\ Sazba dokumentů a matematických výrazů\\}
    
\vspace{\stretch{0.618}}

\end{center}

{\LARGE 2019 \hfill Vojtěch Jurka (xjurka08)}

\newpage
\twocolumn

\setcounter{page}{1}
\label{page}

\section*{Úvod}
V této úloze si vyzkoušíme sazbu titulní strany, matematic\-kých vzorců, prostředí a dalších textových struktur obvyklých pro technicky zaměřené texty (například rovnice (\ref{eq1}) nebo Definice \ref{def1} na straně \pageref{page}). Pro odkazovaní na vzorce a~struktury zásadně používáme příkaz \verb|\label| a \verb|\ref| případně \verb|\pageref| pokud se chceme odkázat na stranu výskytu.

Na titulní straně je využito sázení nadpisu podle optického středu s využitím zlatého řezu. Tento postup byl probírán na přednášce. Dále je použito odřádkování se zadanou relativní velikostí 0.4\,em a 0.3\,em.

\section{Matematický text} 

Nejprve se podíváme na sázení matematických symbolů a~výrazů v plynulém textu včetně sazby definic a vět s využitím balíku \verb|amsthm|. Rovněž použijeme poznámku pod čarou s použitím příkazu \verb|\footnote|. Někdy je vhodné použít konstrukci \verb|\mbox{}|, která říká, že text nemá být zalomen.

\begin{definition}
\label{def1}
Zásobníkový automat \textit{(ZA) je definován jako sedmice tvaru $A= (Q,\Sigma,\Gamma,\delta,q_0,Z_0,F)$, kde:}
\begin{itemize}
    \item \textit{Q je konečná množina} vnitřních (řídicích) stavů,
    \item \textit{$\Sigma$ je konečná} vstupní abeceda,
    \item \textit{$\Gamma$ je konečná} zásobníková abeceda,
    \item \textit{$\delta$ je} přechodová funkce $Q\times(\Sigma\cup\{\epsilon\})\times\Gamma\rightarrow 2^{Q\times\Gamma^*}$,
    \item $q_0\in Q$ počáteční stav, $Z_0\in\Gamma$ je startovací symbol zásobníku a $F\subseteq Q$ je množina koncových stavů.
\end{itemize}

Nechť $P=(Q,\Sigma,\Gamma,\delta,q_0,Z_0,F)$ je zásobníkový auto\-mat. \textit{Konfigurací} nazveme trojici $(q,w,\alpha)\in Q\times\Sigma^*\times\Gamma^*$, kde $q$ je aktuální stav vnitřního řízení, $w$ je dosud nezpra\-covaná část vstupního řetězce a $\alpha=Z_{i_1}Z_{i_2}\ldots Z_{i_k}$ je obsah zásobníku\footnote{$Z_{i_1}$ je vrchol zásobníku}.

\end{definition}

\subsection{Podsekce obsahující větu a odkaz}

\begin{definition}
\label{def2}
Řetězec $w$ nad abecedou $\Sigma$ je přijat ZA $A$ \textit{jestliže $(q_0,w,Z_0) \underset{A}{\overset{*}{\vdash}} (q_F,\epsilon,\gamma)$ pro nějaké $\gamma\in\Gamma^*$ a $q_F\in F$. Množinu $L(A)=\{w\,|\, w$ je přijat ZA $A\}\subseteq \Sigma^*$ nazýváme} jazyk přijímaný TS $M$.
\end{definition}

Nyní si vyzkoušíme sazbu vět a důkazů opět s použitím balíku \verb|amsthm|.

\begin{theorem}
Třída jazyků, které jsou přijímány ZA, odpovídá bezkontextovým jazykům.
\end{theorem}

\begin{proof}
V důkaze vyjdeme z Definice \ref{def1} a \ref{def2}.
\end{proof}

\section{Rovnice a odkazy}

Složitější matematické formulace sázíme mimo plynulý text. Lze umístit několik výrazů na jeden řádek, ale pak je třeba tyto vhodně oddělit, například příkazem \verb|\quad|.
$$\sqrt[i]{x_i^3}\quad \text{kde}\ x_i\ \text{je}\ i\text{-té sudé číslo splňující}\quad x_i^{2-{x_i^i}^2}\leq x_i^{y_i^3}$$

V rovnici (\ref{eq1}) jsou využity tři typy závorek s různou explicitně definovanou velikostí.

\begin{eqnarray}
\label{eq1}
    x &=&{\bigg[{{\Big{\{}[a+b]*c\Big{\}}}^d}\ominus 1\bigg]}^{1/2} \\
    y&=&\lim_{x \to \infty} \frac{\frac{1}{\log_{10} x}}{\sin^2 + \cos^2 x}
    \nonumber
\end{eqnarray}

V této větě vidíme, jak vypadá implicitní vysázení li-mity $\lim_{n\to \infty} f(n)$ v normálním odstavci textu. Podobně je to i s dalšími symboly jako $\prod_{i=a}^n 2^i$ či $\bigcap_{A\in B} A$. V~případě vzorců $\lim\limits _{n \to \infty} f(n)$ a $\prod\limits _{i=1}^n 2^i$ jsme si vynutili méně úspornou sazbu příkazem\verb|\limits|.

\begin{eqnarray}
    \int_b^a g(x)\,\mathrm{d}x &=& - \int\limits_a^b f(x)\,\mathrm{d}x\\
    \overline{\overline{A\wedge B}} &\Leftrightarrow& \overline{\overline{A} \vee \overline{B}}
\end{eqnarray}

\section{Matice}

Pro sázení matic se velmi často používá prostředí \verb|array| a závorky (\verb|\left|, \verb|\right|).

$$\left[
\begin{array}{ccc}
&\widehat{\beta + \gamma} & \hat{\pi} \\
\vec{a} & \overleftrightarrow{AC} &
\end{array} 
\right] 1\Longleftrightarrow {\mathbb Q} = \mathbf{R}
$$
$$
\mathbf{A}=
\left|
\begin{array}{cccc}
a_{11}&a_{12}&\ldots&a_{1n}\\
a_{21}&a_{22}&\ldots&a_{2n}\\
\vdots&\vdots&\ddots&\vdots\\
a_{m1}&a_{m2}&\ldots&a_{mn}\\
\end{array}
\right|
=
\begin{array}{rl}
    t & u \\
    v & w
\end{array}
=tw-uv
$$

Prostředí \verb|array| lze úspěšně využít i jinde.
$$
\left(
\begin{array}{c}
n\\
k
\end{array}
\right)
=
 \left\{
 \begin{array}{ll}0 & \text{pro } x<0 \text{ nebo }k>n\\
 \frac{n!}{k!(n-k)!} & \text{pro } 0 \leq k \leq n
 \end{array}
 \right.
$$



\end{document}
